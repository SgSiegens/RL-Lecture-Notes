\section{Distributional Reinforcement Learning}
Distributional Reinforcement Learning (DRL) is an extension of traditional reinforcement 
learning (RL) that focuses on modeling the distribution of return values (rewards) rather 
than just predicting the expected return.\newline
In traditional RL, the goal is often to maximise the expected cumulative reward (value function) for a given state . $$V ^\pi(s) = \mathbb{E}_\pi\left[\sum_{k=0}^\infty\gamma^k r_{t+1+k}|s_t=s\right]$$
However, in distributional reinforcement learning, instead of focusing on a single expected 
value, we learn the full distribution of the returns. This means that, instead of predicting 
just the expected return (like a scalar value), we model the entire range of possible future 
rewards and their probabilities. \newline
Imagine you're playing a video game where, at some points, you can make a risky move that 
might either yield a high reward or result in a penalty. In traditional RL, you might expect 
the average reward (e.g., 5 points), but in distributional RL, you can learn the probability 
distribution of rewards—e.g., 70\% chance for +10 points, 20\% chance for +5 points, and 
10\% chance for -5 points. This distributional approach helps the agent better weigh its 
decisions.

\subsection{Resources}
The content presented here is intended as a brief overview and will be expanded and improved in the future. For a deeper 
understanding, a highly recommended lecture is ''Deep RL Bootcamp Frontiers Lecture I: Recent Advances, Frontiers, and 
Future of Deep RL`` \cite{DeepRlBootcamp}.




